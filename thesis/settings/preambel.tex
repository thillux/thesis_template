\usepackage{amsmath}
\usepackage{amssymb}

%%%%%%%%%%%%%%%%%%%%%%%%%%%%%%%%%%

\usepackage{bm}
\usepackage[nice]{nicefrac}     % schräge Bruchstriche
\usepackage{algorithm}
\usepackage{algpseudocode}

\floatname{algorithm}{Algorithmus}
\renewcommand{\listalgorithmname}{Algorithmenverzeichnis}

%%%%%%%%%%%%%%%%%%%%%%%%%%%%%%%%%%

\usepackage{bookmark}                           % Einstellungen für hyperref
\usepackage{booktabs}                           % nice tables
\usepackage[hang,flushmargin]{footmisc}         % schönere Fußzeile
\usepackage{paralist}                           % neue Listen
\usepackage{graphicx}                           % Einbinden von Abbildungen
\usepackage{xcolor}                             % Farben
\usepackage{lipsum}
\usepackage{wrapfig}
\usepackage[printonlyused]{acronym}             % Abbkürzungsverzeichnis
\usepackage{blindtext}
\usepackage{texnames}
\usepackage{subcaption}
\usepackage{ccicons}							% Creative Commons Icons
\usepackage{nameref}							% ermöglicht Referenzierung von Abschnittstiteln

\usepackage[style=ieee]{biblatex}
\addbibresource{bib/books.bib}
\addbibresource{bib/papers.bib}
\addbibresource{bib/rfcs.bib}
\addbibresource{bib/websources.bib}

\renewcommand{\bibname}{Literaturverzeichnis}

%%%%%%%%%%%%%%%%%%%%%%%%%%%%%%%%%%

\DeclareMathOperator*{\argmin}{arg\,min}
\DeclareMathOperator*{\argmax}{arg\,max}

\usepackage{esvect}

%%%%%%%%%%%%%%%%%%%%%%%%%%%%%%%%%%

\usepackage{multicol}
\setlength{\columnsep}{1em}

%%%%%%%%%%%%%%%%%%%%%%%%%%%%%%%%%%

\usepackage{scrlayer-scrpage}
\pagestyle{scrheadings}
\chead[]{}
\automark[chapter]{section}

%%%%%%%%%%%%%%%%%%%%%%%%%%%%%%%%%%

\usepackage{hyperref}
\makeatletter
\hypersetup{
    unicode=true,                               % non-Latin characters in Acrobat’s bookmarks
    colorlinks=true,                            % false: boxed links; true: colored links
    linkcolor=black,                            % color of internal links
    citecolor=black,                            % color of links to bibliography
    filecolor=black,                            % color of file links
    urlcolor=black,                             % color of external links
    pdftoolbar=true,                            % show Acrobat’s toolbar?
    pdfmenubar=true,                            % show Acrobat’s menu?
    pdffitwindow=false,                         % window fit to page when opened
    pdfstartview={FitH},                        % fits the width of the page to the window
    pdfnewwindow=true,                          % links in new window
    pdftitle={\@title},
    pdfsubject={\TypDerArbeit},
    pdfauthor={\@author},
    pdfkeywords={\Schlagworte}
}
\makeatother

%%%%%%%%%%%%%%%%%%%%%%%%%%%%%%%%%%

% http://html5up.net/read-only
\definecolor{thillux-green}{HTML}{4ACAA8}

\definecolor{thillux-red}{HTML}{E7746F}

\definecolor{thillux-grey}{HTML}{989898}

\definecolor{thillux-blue}{HTML}{0E94EC}

\usepackage{tikz}
\usetikzlibrary{shapes,snakes}

\tikzset{
	% Styling of header text is done using key/value options for TikZ nodes. See
	% section 16.4 of the PGF manual for a complete list of options that affect
	% text.
	headings/base/.style = {
		% Zap node seperation, set text width and alignment.
		outer sep = 0pt,
		% Trim off 2/3rd of an em to compensate for the inner xsep which spaces the
		% text nicely away from the left side, but causes the node to hang into the
		% right margin.
		text width = {\columnwidth - 0.6666em},
		align = left
	},
	headings/chapter/.style = {
		outer sep = 0pt,
		inner sep = 0pt,
		font = \huge,
		text = thillux-blue,
		align = left,
		text width = \columnwidth
	},
	headings/section/.style = {
		headings/base,
		fill = thillux-blue,
		font = \large,
		minimum height = (1.6em),
		rectangle,
		rounded corners,
		text = white,
	},
	headings/subsection/.style = {
		headings/base,
		fill = thillux-grey!65,
		font = \large,
		minimum height = (1.6em),
		rectangle,
		rounded corners,
		text = white
	}
}
\newcommand{\colorboxedsec}[2]{
	\tikz{\node[headings/#1]{#2};}}

\setkomafont{chapter}{\colorboxedsec{chapter}}
\setkomafont{section}{\colorboxedsec{section}}
\setkomafont{subsection}{\colorboxedsec{subsection}}

%%%%%%%%%%%%%%%%%%%%%%%%%%%%%%%%%%

\setlength{\parindent}{0pt}

%%%%%%%%%%%%%%%%%%%%%%%%%%%%%%%%%%
\newcounter{anforderung}
\newcommand{\anfl}[1]{%
\refstepcounter{anforderung}%
(A\arabic{anforderung})%
\label{anf:#1}%
}

\usepackage[toc,page]{appendix}

\usepackage{enumitem}
\setlist[itemize]{leftmargin=2.5em}

\newcommand{\itemizeBox}{\hspace{1em}\raisebox{0.8ex}{\fcolorbox{thillux-blue}{thillux-blue!35}{}}}
\newcommand{\itemizeArrow}{%
	\raisebox{-0.1ex}{%
		\resizebox{0.5em}{0.9em}{%
			\begin{tikzpicture}%
			\draw [fill=none,line width=20pt, color=thillux-red] (0,1)--(1,0)--(0,-1);%
			\end{tikzpicture}%
		}%
	}%
}

\setlist[enumerate]{label=(\roman*)}

\renewcommand{\labelitemi}{\itemizeArrow}
\renewcommand{\labelitemii}{\itemizeBox}

%%%%%%%%%%%%%%%%%%%%%%%%%%%%%%%%%%%%%%%%%%%%%%%%

% example boxes

\usetikzlibrary{calc}

\newcommand{\TXExample}[1]{
\begin{center}%
	\begin{tikzpicture}%
		\node [draw=none, fill=thillux-grey!35, very thick,
		rectangle, inner sep=10pt, inner ysep=10pt] (box){%
			\begin{minipage}{0.85\textwidth}%
				#1%
			\end{minipage}%
		};%
		\draw[line width=1.5ex, thillux-blue] ($(box.south west) + (0,0.6pt)$) |- ($(box.north west) - (0,0.6pt)$);%
	\end{tikzpicture}%
\end{center}%
}

\newcommand{\TXExampleWithTitle}[2]{
	\begin{center}%
		\begin{tikzpicture}%
		\node [draw=none, fill=thillux-grey!35, very thick,
		rectangle, inner sep=10pt, inner ysep=10pt] (box){%
			\begin{minipage}{0.85\textwidth}%
			#2%
			\end{minipage}%
		};%
		\node [draw=none, fill=thillux-grey, very thick,
		rectangle, inner sep=10pt, inner ysep=8pt] at ($(box.north) + (0,10.8pt)$) (boxtitle) {%
			\begin{minipage}{0.85\textwidth}%
			\color{white}\textbf{#1}%
			\end{minipage}%
		};
		\draw[line width=1.5ex, thillux-blue] ($(box.south west) + (0,0.6pt)$) |- ($(boxtitle.north west) - (0,0.6pt)$);
		\end{tikzpicture}%
	\end{center}%
}

\usepackage{relsize}
